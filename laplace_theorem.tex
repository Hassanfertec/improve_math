Vamos aprimorar a demonstração para tornar a ideia de $S_n$ mais compreensível. Suponha que $B$ seja uma matriz $n \times n$ e $i,j \in \{1,2,\dots ,n\}$. Para clareza, também rotulamos as entradas de $B$ que compõem sua matriz menor $i,j$, denotada por $M_{ij}$, como $(a_{st})$ para $1 \leq s,t \leq n-1$.

Considere os termos na expansão de $|B|$ que têm $b_{ij}$ como um fator. Cada um tem a forma

$$\operatorname{sgn} \tau \, b_{1,\tau (1)}\cdots b_{i,j}\cdots b_{n,\tau (n)} = \operatorname{sgn} \tau \, b_{ij} a_{1,\sigma (1)}\cdots a_{n-1,\sigma (n-1)}$$

para alguma permutação $\tau \in S_n$ com $\tau (i) = j$, e uma permutação única e evidentemente relacionada $\sigma \in S_{n-1}$ que seleciona as mesmas entradas menores que $\tau$. Da mesma forma, cada escolha de $\sigma$ determina um $\tau$ correspondente, ou seja, a correspondência $\sigma \leftrightarrow \tau$ é uma bijeção entre $S_{n-1}$ e $\{\tau \in S_{n}: \tau (i) = j\}$. Usando a notação de duas linhas de Cauchy, a relação explícita entre $\tau$ e $\sigma$ pode ser escrita como

$$\sigma = \begin{pmatrix} 1 & 2 & \cdots & i & \cdots & n-1 \\ (\leftarrow )_{j}(\tau (1)) & (\leftarrow )_{j}(\tau (2)) & \cdots & (\leftarrow )_{j}(\tau (i+1)) & \cdots & (\leftarrow )_{j}(\tau (n)) \end{pmatrix}$$

onde $(\leftarrow )_{j}$ é uma notação temporária para um ciclo $(n,n-1,\cdots ,j+1,j)$. Esta operação decrementa todos os índices maiores que $j$ para que cada índice se encaixe no conjunto $\{1,2,...,n-1\}$.

A permutação $\tau$ pode ser derivada de $\sigma$ da seguinte maneira. Defina $\sigma' \in S_{n}$ por $\sigma' (k) = \sigma (k)$ para $1 \leq k \leq n-1$ e $\sigma' (n) = n$. Então $\sigma'$ é expresso como

$$\sigma' = \begin{pmatrix} 1 & 2 & \cdots & i & \cdots & n-1 & n \\ (\leftarrow )_{j}(\tau (1)) & (\leftarrow )_{j}(\tau (2)) & \cdots & (\leftarrow )_{j}(\tau (i+1)) & \cdots & (\leftarrow )_{j}(\tau (n)) & n \end{pmatrix}$$

Agora, a operação que aplica $(\leftarrow )_{i}$ primeiro e depois aplica $\sigma'$ é (observe que aplicar A antes de B é equivalente a aplicar o inverso de A à linha superior de B na notação de duas linhas)

$$\sigma' (\leftarrow )_{i} = \begin{pmatrix} 1 & 2 & \cdots & i+1 & \cdots & n & i \\ (\leftarrow )_{j}(\tau (1)) & (\leftarrow )_{j}(\tau (2)) & \cdots & (\leftarrow )_{j}(\tau (i+1)) & \cdots & (\leftarrow )_{j}(\tau (n)) & n \end{pmatrix}$$

onde $(\leftarrow )_{i}$ é uma notação temporária para um ciclo $(n,n-1,\cdots ,i+1,i)$.

A operação que aplica $\tau$ primeiro e depois aplica $(\leftarrow )_{j}$ é

$$(\leftarrow )_{j}\tau = \begin{pmatrix} 1 & 2 & \cdots & i & \cdots & n-1 & n \\ (\leftarrow )_{j}(\tau (1)) & (\leftarrow )_{j}(\tau (2)) & \cdots & n & \cdots & (\leftarrow )_{j}(\tau (n-1)) & (\leftarrow )_{j}(\tau (n)) \end{pmatrix}$$

As duas operações acima são iguais, portanto,

$$(\leftarrow )_{j}\tau = \sigma' (\leftarrow )_{i}$$
$$\tau = (\rightarrow )_{j}\sigma' (\leftarrow )_{i}$$

onde $(\rightarrow )_{j}$ é o inverso de $(\leftarrow )_{j}$ que é $(j,j+1,\cdots ,n)$.

Assim,

$$\tau = (j,j+1,\ldots ,n)\sigma' (n,n-1,\ldots ,i)$$

Como os dois ciclos podem ser escritos respectivamente como $n-i$ e $n-j$ transposições,

$$\operatorname{sgn} \tau = (-1)^{2n-(i+j)}\operatorname{sgn} \sigma' = (-1)^{i+j}\operatorname{sgn} \sigma .$$

E como o mapa $\sigma \leftrightarrow \tau$ é bijetivo,

$$\sum _{i=1}^{n}\sum _{\tau \in S_{n}:\tau (i)=j}\operatorname{sgn} \tau \,b_{1,\tau (1)}\cdots b_{n,\tau (n)} = \sum _{i=1}^{n}\sum _{\sigma \in S_{n-1}}(-1)^{i+j}\operatorname{sgn} \sigma \,b_{ij}a_{1,\sigma (1)}\cdots a_{n-1,\sigma (n-1)} = \sum _{i=1}^{n}b_{ij}(-1)^{i+j}\sum _{\sigma \in S_{n-1}}\operatorname{sgn} \sigma \,a_{1,\sigma (1)}\cdots a_{n-1,\sigma (n-1)} = \sum _{i=1}^{n}b_{ij}(-1)^{i+j}M_{ij}$$

A partir do qual o resultado segue. Da mesma forma, o resultado se mantém se o índice da soma externa fosse substituído por $j$.
